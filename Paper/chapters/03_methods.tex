\section{Conclusion}\label{sec:Ecotoxicology}
As we can see from the case study, the simulated precipitation in the city is 708mm, much lower compared to the statistic data 852.7mm, so the simulation is not that precise. However, this model only includes the simple dynamics and microphysical processes, but topography, vegetation, and other processes may influence the precipitation. That may explain the difference of the precipitation. Also meteological conditions may influcence it. The data I used in the model, e.g. RH, temperature, are the annual mean data. And 708mm is from the simulated daily precipitation multiplying the numebr of raining days in a year. In the model, topography is simplified, I only change the height of the peak and the width of the mountain. So those reasons may explain the difference.

I understand the content of the tutorial, and I want to do more complicated case study. For example, I should have implemented more detailed topography in Qin Mountains Region, more detailed climate data and compared the simulated data with the longer time period climate data of Han Zhong, which I can only get in China.  I know case study is not that quantitatively, but I have learned what should be included in a meteological model, how to run the model and how to understand the weather processes with the help of the model from the tutorial. 

\section{Acknowledgement}
I hereby should thank the tutors in the tutorial class for helping me with the code, explaining me about the processes included in the model and providing the exercise sheets on the website. Also, I should thank Ari, Annika, Ryan and Prisco for cooperation. 
















